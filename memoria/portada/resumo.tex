%%%%%%%%%%%%%%%%%%%%%%%%%%%%%%%%%%%%%%%%%%%%%%%%%%%%%%%%%%%%%%%%%%%%%%%%%%%%%%%%

\pagestyle{empty}
\begin{abstract}
    El LiDAR (Light Detection and Ranging) es una tecnología de detección remota que emplea pulsos láser para medir distancias y generar modelos 3D de objetos y entornos. En el contexto de la detección de árboles, esta destaca por su capacidad para capturar detalles como la estructura y ubicación de la vegetación. En los últimos años, esta metodología de estudio de los bosques ha ganado popularidad. En este trabajo, se plantea y desarrolla la aplicación de algoritmos basados en características geométricas de la vegetación, junto con el análisis de los resultados obtenidos a partir de modelos de datos LIDAR de Luxemburgo.

  
  \vspace*{25pt}
  \begin{segundoresumo}
    LiDAR (Light Detection and Ranging) is a remote sensing technology that employs laser pulses to measure distances and generate three-dimensional models of objects and environments. In the context of tree detection, LIDAR stands out for its ability to capture precise details about the structure and location of vegetation. In recent years, this approach to studying forests has gained popularity. In this work, the application of algorithms based on geometric characteristics of vegetation is proposed and developed, along with the analysis of the results obtained from LIDAR data models from Luxembourg.

  \end{segundoresumo}
\vspace*{25pt}
\begin{multicols}{2}
\begin{description}
\item [\palabraschaveprincipal:] \mbox{} \\[-20pt]
    \begin{itemize}
        \item LiDAR
        \item Nube de Puntos
        \item Arboles
        \item Análisis de datos 
        \item Características geométricas
        \item Dosel arbóreo
        \item Gestión forestal
    \end{itemize} 
\end{description}
\begin{description}
\item [\palabraschavesecundaria:] \mbox{} \\[-20pt]
    \begin{itemize}
        \item LiDAR
        \item Point Cloud
        \item Trees
        \item Data Analysis
        \item Geometric Features
        \item Tree Canopy
        \item Forest Management
    \end{itemize}
\end{description}
\end{multicols}

\end{abstract}
\pagestyle{fancy}

%%%%%%%%%%%%%%%%%%%%%%%%%%%%%%%%%%%%%%%%%%%%%%%%%%%%%%%%%%%%%%%%%%%%%%%%%%%%%%%%
