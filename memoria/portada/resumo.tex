%%%%%%%%%%%%%%%%%%%%%%%%%%%%%%%%%%%%%%%%%%%%%%%%%%%%%%%%%%%%%%%%%%%%%%%%%%%%%%%%

\pagestyle{empty}
\begin{abstract}
    El LiDAR (Light Detection and Ranging) es una tecnología de detección remota que emplea pulsos láser para medir distancias y generar modelos 3D de objetos y entornos. En el contexto de la detección de árboles, esta destaca por su capacidad para capturar detalles como la estructura y ubicación de los árboles . En los últimos años, esta metodología de estudio de los bosques ha ganado popularidad. En este trabajo, se plantea y desarrolla la aplicación de algoritmos basados en características geométricas de los arboles para ser capaces de detectar y contabilizarlos, junto a esto realizaremos un análisis de los resultados obtenidos. 

  
  \vspace*{25pt}
  \begin{segundoresumo}
   LiDAR (Light Detection and Ranging) is a remote sensing technology that uses laser pulses to measure distances and generate 3D models of objects and environments. In the context of tree detection, it stands out for its ability to capture details such as the structure and location of trees. In recent years, this methodology for studying forests has gained popularity. In this work, we propose and develop the application of algorithms based on the geometric features of trees to be able to detect and count them, and alongside this, we will conduct an analysis of the obtained results.

  \end{segundoresumo}
\vspace*{25pt}
\begin{multicols}{2}
\begin{description}
\item [\palabraschaveprincipal:] \mbox{} \\[-20pt]
    \begin{itemize}
        \item LiDAR
        \item Nube de Puntos
        \item Arboles
        \item Análisis de datos 
        \item Características geométricas
        \item Dosel arbóreo
        \item Gestión forestal
    \end{itemize} 
\end{description}
\begin{description}
\item [\palabraschavesecundaria:] \mbox{} \\[-20pt]
    \begin{itemize}
        \item LiDAR
        \item Point Cloud
        \item Trees
        \item Data Analysis
        \item Geometric Features
        \item Tree Canopy
        \item Forest Management
    \end{itemize}
\end{description}
\end{multicols}

\end{abstract}
\pagestyle{fancy}

%%%%%%%%%%%%%%%%%%%%%%%%%%%%%%%%%%%%%%%%%%%%%%%%%%%%%%%%%%%%%%%%%%%%%%%%%%%%%%%%
