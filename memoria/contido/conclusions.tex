\chapter{Conclusións}
\label{chap:conclusions}

\lettrine{E}{n} este ultimo capitulo se expondrán las conclusiones obtenidas tras realizar este proyecto y los aprendizajes realizados. También Comentaremos que posibles lineas de futuro se podrían seguir para mejorar el trabajo realizado hasta ahora.

\section{Conclusiones}


\section{Trabajo Futuro}
En este proyecto, se emplearon técnicas geométricas para determinar la presencia de árboles. Durante la fase inicial, se consideró la utilización de una red neuronal como \textit{PointNet} \cite{pointnet} para esta tarea. Sin embargo, esta aproximación presentaba un desafío en términos de la necesidad de una amplia y diversa base de datos de miles de árboles de distintas especies, con el fin de asegurar la robustez del sistema. Debido a los plazos ajustados del proyecto, se optó por descartar esta opción.

En relación al rendimiento del código desarrollado, una mejora potencial consistiría en lograr compatibilidad con librerías GPU como \textit{CUDA} o \textit{OpenCL}. Esto permitiría aprovechar la capacidad de procesamiento gráfico para ejecutar en paralelo las numerosas operaciones necesarias en el procesamiento de nubes de puntos. Otra forma de ganar rendimiento seria portar el código a Rust un lenguaje mucho mas rápido y eficiente que Python.